\documentclass[t, 10pt]{beamer}
%%\documentclass[t,handout]{beamer}

\usepackage{graphicx}
\usepackage{epsfig}
\usepackage{psfrag}
\usepackage[english]{babel}
\usepackage{color}
%Mathematics packages
\usepackage{amsmath}
\usepackage{mathrsfs}
\usepackage{amsfonts}

\usepackage{enumerate}


\graphicspath{{./images/}} % Figures path - used in graphicx

\selectcolormodel{cmyk}

\mode<presentation>

%THEMES - Please refer to these chapters in the beamer documentation.
% Presentation themes : Chapter 15
% Color themes : Chapter 17
% Font themes : Chapter 18


\usetheme{Pittsburgh}
\usecolortheme{orchid}
\usefonttheme{default}

%---------------------------Title frame definition------------------------------------- 

\title{Usage Management of Personal Medical Records}
\author [Chris]{Christopher Lamb, Pramod Jamkhedkar, and Gregory Heileman}
\institute[University of New Mexico]{
\inst {}Department of Electrical and Computer Engineering\\
University of New Mexico}
\date{Februrary 24, 2011}
\titlegraphic{
\begin{figure} 
\includegraphics[width = 7cm]{UNM}
\end{figure}}

% Delete this, if you do not want the table of contents to pop up at
% the beginning of each subsection:
%\AtBeginSubsection[]
%{
%  \begin{frame}<beamer>
%    \frametitle{Outline}
%     \tableofcontents[currentsection,currentsubsection]
%  \end{frame}
%}


\begin{document}

\begin{frame}
\titlepage
\end{frame}

% This command will make the logo appear on all frames excluding the title frame.
\logo {\includegraphics[width = 2.5cm]{UNM}}

\begin{frame}
\frametitle{Outline}
\tableofcontents 
\end{frame}

\begin{frame}
\frametitle{Introduction}

\end{frame}

\section{UNM Informatics}
\begin{frame}
\frametitle{Areas of Study}
\begin{itemize}
\item \textit{UNM Informatics}: Information security, theory, and architectures this work is specific to information security 
\pause
\item \textit{Usage Management}: Control of how an artifact is used, covering everything \textit{after} access
\end{itemize}
\pause
Some quick definitions:
\begin{itemize}
\item \textit{PMR}: Personal medical record; in this case, this record is electronic
\item \textit{UM}: Usage management
\end{itemize}
\end{frame}

\section{Personal Medical Records}
\begin{frame}
\frametitle{UM and PMRs}

We believe PMRs have certain attributes that aren't addressed well by current management systems:
\pause
\begin{itemize}
\item \textit{Mashable}
\pause
\item \textit{Controllable}
\pause
\item \textit{Available}
\end{itemize}
\pause

Usage Management of PMRs enables these things, providing fine-grained management of \textit{information contained in PMRs}
\newline
\newline
\pause

This opens new business models:
\pause
\begin{itemize}
\item \textit{Remote Access}
\pause
\item \textit{Monitoring}
\pause
\item \textit{Custom Care}
\pause
\item \textit{Data Marketplace}
\end{itemize}
 
\end{frame}

\section{UM Primer}
\begin{frame}
\frametitle{UM Primer - UM System}

Three basic things:
\begin{figure}
\includegraphics[width = 6cm]{Integrated}
\end{figure}

\begin{itemize}
\item \textit{Usage Management Mechanism}
\item \textit{Policy}
\item \textit{Context}
\end{itemize}

\end{frame}

\begin{frame}
\frametitle{UM Primer - Ontology}

Ontology of domain required to pull it all together
\begin{figure}
\includegraphics[width = 6cm]{UMOntology}
\end{figure}

\end{frame}

\section{Data Marketplace}
\begin{frame}
\frametitle{Data Marketplace - Ontology}
\pause

\begin{figure}
\includegraphics[width = 6cm]{ontology}
\end{figure}
\pause

\begin{itemize}
\item Use a combination of static and dynamic policy evaluation
\begin{itemize}
\item Static filtering of records pre-distribution is more efficient
\end{itemize}
\pause

\item Note relationships to previous domain ontology
\begin{itemize}
\item Common elements include \textbf{Record} entities and specific roles
\end{itemize}
\end{itemize}

\end{frame}

\begin{frame}
\frametitle{Data Marketplace - Roles}

\begin{figure}
\includegraphics[width = 6cm]{roles}
\end{figure}
\pause

In General:
\begin{itemize}
\item \textit{Producers} produce data, \textit{Consumers} directly consume or redistribute data
\end{itemize}

\end{frame}

\begin{frame}
\frametitle{Data Marketplace - System Attributes}

\begin{itemize}

\item \textit{Editability} Certain fields of that record should be editable by the owner.  Other fields must only be editable by specific medical providers.

\item \textit{Roles} Verifiable roles related to ownership of specific areas of a given record.

\item \textit{Auditability} Keeps a clear record of who edited what, what those specific changes were, how they were made, and when. 

\item \textit{Security} Use of modern security systems as much as possible to provide additional control over assets. 

\item \textit{Accessibility} Wide accessibility geographically, access to medical information from devices with a variety of form factors.

\item \textit{Performance} Core functionality must be high performance.

\item \textit{Flexibility} This system and the data it manages can be used in a wide variety of contexts.

\item \textit{Extensibility} It must provide programmatic interfaces.

\end{itemize}

\end{frame}

\begin{frame}
\frametitle{Data Marketplace - Logical View}

\begin{figure}
\includegraphics[width = 6cm]{HisbSystemArch}
\end{figure}

\end{frame}

\begin{frame}
\frametitle{Conclusions}

\begin{itemize}
\item PMRs need some kind of usage management
\end{itemize}

\end{frame}

%\bibliographystyle{plain}
%\bibliography{drm}

\end{document}


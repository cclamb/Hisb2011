
\documentclass[10pt, conference, compsocconf]{IEEEtran}

\ifCLASSINFOpdf
  \usepackage[pdftex]{graphicx}
  % declare the path(s) where your graphic files are
  \graphicspath{{./images/}}
  % and their extensions so you won't have to specify these with
  % every instance of \includegraphics
  \DeclareGraphicsExtensions{.pdf,.jpeg,.png,.jpg}
\else
  % or other class option (dvipsone, dvipdf, if not using dvips). graphicx
  % will default to the driver specified in the system graphics.cfg if no
  % driver is specified.
  \usepackage[dvips]{graphicx}
  % declare the path(s) where your graphic files are
  \graphicspath{{./images/}}
  % and their extensions so you won't have to specify these with
  % every instance of \includegraphics
  \DeclareGraphicsExtensions{.eps}
\fi

% correct bad hyphenation here
\hyphenation{op-tical net-works semi-conduc-tor}


\begin{document}
%
% paper title
% can use linebreaks \\ within to get better formatting as desired
\title{Usage Management of Electronic Medical Records}


% author names and affiliations
% use a multiple column layout for up to two different
% affiliations
\author{\IEEEauthorblockN{Christopher C. Lamb, Pramod A. Jamkhedkar, Gregory L. Heileman, Ravi Kadaboina}
\IEEEauthorblockA{University of New Mexico\\
Department of Electrical and Computer Engineering\\
Albuquerque, NM 87131-0001 \\
\{cclamb, pramod54, heileman, ravik\}@ece.unm.edu}
}

% conference papers do not typically use \thanks and this command
% is locked out in conference mode. If really needed, such as for
% the acknowledgment of grants, issue a \IEEEoverridecommandlockouts
% after \documentclass

% for over three affiliations, or if they all won't fit within the width
% of the page, use this alternative format:
% 
%\author{\IEEEauthorblockN{Michael Shell\IEEEauthorrefmark{1},
%Homer Simpson\IEEEauthorrefmark{2},
%James Kirk\IEEEauthorrefmark{3}, 
%Montgomery Scott\IEEEauthorrefmark{3} and
%Eldon Tyrell\IEEEauthorrefmark{4}}
%\IEEEauthorblockA{\IEEEauthorrefmark{1}School of Electrical and Computer Engineering\\
%Georgia Institute of Technology,
%Atlanta, Georgia 30332--0250\\ Email: see http://www.michaelshell.org/contact.html}
%\IEEEauthorblockA{\IEEEauthorrefmark{2}Twentieth Century Fox, Springfield, USA\\
%Email: homer@thesimpsons.com}
%\IEEEauthorblockA{\IEEEauthorrefmark{3}Starfleet Academy, San Francisco, California 96678-2391\\
%Telephone: (800) 555--1212, Fax: (888) 555--1212}
%\IEEEauthorblockA{\IEEEauthorrefmark{4}Tyrell Inc., 123 Replicant Street, Los Angeles, California 90210--4321}}

% make the title area
\maketitle


\begin{abstract}
Electronic medical record management is under new scruitiny as private companies move into the market and government agencies actively address percieved health care distribution inequalities and inefficiencies.  Current systems are coarse-grained and provide consumers very little actual control over their data.  Herein, we propose an alternative system for managing the use of healthcare infomormation.  This system is more granular, allows for data mining and repackaging, and gives users more control over data while allowing said data to be distributed as much as needed.  In this paper, we outline the characteristics of such a system, present relavant background information and research leading to the system design, and cover two specific use scenarios supported by this system that are difficult to control using simpler access control strategies.
\end{abstract}

\begin{IEEEkeywords}
%component; formatting; style; styling;
usage management; electronic medical records

\end{IEEEkeywords}


% For peer review papers, you can put extra information on the cover
% page as needed:
% \ifCLASSOPTIONpeerreview
% \begin{center} \bfseries EDICS Category: 3-BBND \end{center}
% \fi
%
% For peerreview papers, this IEEEtran command inserts a page break and
% creates the second title. It will be ignored for other modes.
\IEEEpeerreviewmaketitle

\section*{Introduction}
% no \IEEEPARstart
New healthcare legislation has spurred previously unknown levels of public and private investment into technologies supporting more efficient healthcare delivery \cite{Emr:Web:Recovery}.   An active area of examination is electronic health records.  Current systems, like Microsoft HealthVault and Google Health are a start in this area, but provide rudimentary control over health information, provide consumers with very little actual control of their information, and essentially demand proprietary lockin to these products because of the amount of effort involved with data transfer \cite{Emr:EvaluationHealthInf}.

We propose an open, consumer-centric approach to health information storage and consumption centered around flexible and granular usage management policies.  User empowering systems in this area are needed to allow users control over the information that represents them, and would be in high demand if appropriately designed \cite{Emr:PyAmWaCr}.  We propose to address this need by bundling health information (either entire records or subsets of records) with traceable and aggregateable usage policies controlled by the users themselves.  Users would have the ability to make aspects of their records available to everyone from research institutions looking for historical information for studies, to specific healthcare providers who need specific information to support diagnoses.  Furthermore, institutions would be able to combine information from groups of users and determine dynamically via policy evaluation how that new set of data can be used in a way that complies with all included user policies.  If the combined dataset cannot be used, policies can be analyzed to determine the cause of the policy conflict.

We will propose, design, and demonstrate a system that supports granular management of the data elements of an electronic medical record.  This management will allow users to specify policies over the data itself rather than the entire record in question, providing control over information dissemination.  We will demonstrate this control in three distinct scenarios.  The first will include two distinct parties negotiating over access to specific information contained in a medical record.  If the parties can reach an agreement, the information consumer will be granted access to specific medical data, for an agreed-upon price.  The second demostrates a data broker combining a set of previously acquired medical record data into an aggregate set for research, if the licensure is in fact compliant between all selelected data elements.  Finally, the aggregated data set will be placed back into the market.

This kind of system, allowing users control over their data in ways fostering ease of dissemination, use and reuse, helps users receive better, more targed care, helps providers easly access required information, and allows this kind of data to be more easily examined and mined.  We use established system design principles, used in the develoment of internet-scale networks to create a open flexible system \cite{Al:04,BlCl:01,ClWrSoBr:02}.  We will standardize certain features, such as operational semantics and ontological domains, but otherwise limit the impact of the policy system on data dissemination as much as possible.

% Usage Management info goes here

\subsection*{Previous Work}
Past research applicable to this area includes usage management, digital rights management (DRM), and access control.  Most of the research applicable to the combination of previous arfifacts into a single aggregate artifact comes from the DRM world in particular.  Generally, these expressive languages have been fundamentally based on different types of mathematical logic or formalisms with reasoning capabilities \cite{ArHu:07,BaMi:06,ChCoEtHaJoLa:03,HaWe:04,HaWe:08,PuWe:02,XiBjFu:08}.  This approach, while useful in closed systems, tends to not work as usefully in more open dynamic environments.  This has led to the development of translation mechanisms to address interoperability needs \cite{HeJa:05,PoPrDe:04,ScTaWo:04}.  This translation process is difficult for most policy languages, and in fact infeasible as a result \cite{KoLaMaMi:04,SaShUe:04}.  Alternative approaches have required the use of sophiticated and powerful languages that must be adopted as a universal standard \cite{OMADRM,ODRL-req,Wa:04,XrML-spec}.  This approach inherently limits innovation and flexibility \cite{HeJa:05,JaHe:04,JaHe:08,JaHeMa:06}.

\section*{New Models}
Engineers and futurists have speculated as to the impact of personal medical records for years \cite{Emr:Web:BestCaseEMR,Emr:Web:WorstCaseEMR}.  Others have speculated on the institutional use of personal medical records by organizations in today's regulated medical environment \cite{Emr:doi:10.1056/NEJMc081118}.  Health records, when under the control of the person they address, are no longer controlled by the Health Insurance Portability and Accountability Act (HIPPA), though the companies that manage them on the user's behalf in these cases are regulated in most aspects by the Electronic Communications Privacy Act \cite{Emr:doi:10.1056/NEJMsb0800220}.  In total, These concerns imply certain requirements on robust medical record systems, making use models and record control more complex.  None of the promises or concerns of personal medical records can be realized or mitigated without strong usage management.  With a dependable usage management capability, personal medical records open new horizons in the services landscape for interested adopters.

\subsection*{A Note on Reliability}
%Consumer, physician(s) (primary, emergency, etc.); editability constraints; auditing
In order for PMRs to be effective, they must be actively used by health care providers.  A system with the wrong kinds of editability constraints or auditing capabilities is at risk of remaining unused by an individual's care providers.  Ideally, these kinds of health records would contain the kind of information a physician would include in a patient's chart.  This is information providers are required to maintain for adequate patient treatment.  If this information can be arbitrarily edited however, it loses it's credibility.

In fact, many employer-sponsored monitoring programs may incentivize gold-plating medical histories.  Systems like Virgin HealthMiles are marketing themselves directly to employers as ways to monitor employee health. \cite{Emr:Web:VirginHealthMiles}.  Companies are using Virgin HealthMiles to track employee exercise, and as an incentive to use the product (and get more exercise), are offering additional contributions to employer-sponsored health savings accounts if employees meet certain criteria.  Similar senarios could be right around the corner for personal health management systems, were employers incentivize employees to decrease blood pressure, change diet, or similar kinds of things.  In those situations, the pressure for users to alter their records to reflect the reality their employers want to see will be immense, and many users are likely to resort to embellishing their records as a result.

Once that happens, health care providers can no longer use the records to provide care.

Any system managing these kinds of records must therefore provide mechanisms to certify, if not the accuracy of the provided information, at least the veracity of it.  Care providers must be able to trust the information provided in a given record, and must not be required to shoulder the burden of viewing the record's edit history in order to do so.  This implies a separation of roles between those who can edit the content of a given record, and those who control how the content of that record may be used.

\subsection*{Remote Information Access}
% Travel, Schools, immigration
% Travel, schools, immigration:
% who are the players?
% what do they need access to?
% why?
% when do they need it?
% where are the players physically located?
% how would this happen with PMR? Describe system; use scenario
% address pro/con whereever appropriate


\subsection*{Monitoring}
% Ongoing health monitoring
% who would use this and why?
% what do they want to see?
% when? what are the implications of that?
% how would this happen? is it coersive? describe system; use scenario
% more pro/con

\subsection*{Custom Care}
% Custom pharma
% What exactly is this, and why would users care?
% how would it work?
% who would use it?
% describe system; use scenario
% pro/con

\subsection*{Data Marketplace}
%Wherever Times is specified, Times Roman or Times New Roman may be used. If neither is available on your system, please use the font closest in appearance to Times. Avoid using bit-mapped fonts if possible. True-Type 1 or Open Type fonts are preferred. Please embed symbol fonts, as well, for math, etc.
The system we describe in the following sections incorporates a market to allow users and brokers to profit from the use of electronic medical data released under mutually acceptable terms, where usage policies accompany filtered data for either dynamic or static evaluation.  Usage policies themselves are essentially unlimited in how they describe the use of a specific medical record.

\section*{Sample System - Data Marketplace}
% Go into what the market is and why we need to use one; describe who would use it and their roles; describe when it would be used and how it could be implemented
Here, we incentivize electronic medical record adoption via the use of a data marketplace.  We have three primary categories of users in mind:

\begin{figure}[!t]
\centering
\includegraphics[width=2in]{roles}
\caption{System Roles}
\label{System Roles}
\end{figure}

\begin{itemize}
\item \textit{Data Producers} who produce and market electronic medical information.  This category is generally limited expressly to individual users who require medical care and other related products.
\item \textit{Data Consumers} who directly consume medical information.  This category includes physicians, research institutions, and the like.
\item \textit{Data Brokers} who acquire and remarket medical data from data producers, making that data available in some kind of value-added for to data consumers.  They are a proper subset of data consumers.
\end{itemize}

\textit{Data Producers} would use the medical data market to profit from their personal information.  When negotiating over specifics concerning how their data can be used, they are free to manipulate any aspect of the usage terms prior to a final agreement with a \textit{data consumer}.  The \textit{data consumer} can accept or reject a specific proposal, as can a \textit{data producer}.  A typical negotiation would look something like this:
\begin{enumerate}
\item A \textit{data consumer} searches the marketplace for medical information meeting specific requirements.  This step is a call to a specific search interface in our example, but could be a manual process.
\item The search yields some results.  This proposed system returns a list of contact information of known \textit{data producers} that have data matching the search requirements.
\item The \textit{data consumer} initiates a negotiation for access to specific data.
\begin{enumerate}
\item The \textit{data consumer} contacts the \textit{data producer} and submits and initial proposal.
\item The \textit{data producer} responds to the initial proposal, either be indicating acceptance, rejecting the proposal, or submitting a counter proposal.
\item The \textit{data consumer} is then free to respond with acceptance, rejection, or a counterproposal of her own.
\end{enumerate}
\item Eventually, the negotiation will conclude with the parties having reached an agreement describing access to specific medical data with associated term or having failed to come to mutually acceptable terms with respect to data access.  
\end{enumerate}

Usage terms in a successful conclusion generally describe what the \textit{data consumer} can access, how and for how long, where it may be accessed, and so on.  It will also usually describe some kind of payment for use, which can be based on any arbitrary number of factors such as time, date, location, attribution, or perhaps in combination with other data.

The market implemented in this system is built around JADE, an open source agent develoment framework based on FIPA agent specifications \cite{Emr:Web:Jade,Emr:Web:Fipa}.

\subsection*{System Ontology}
%Describe ontology, describe relationships, define elements; this is a system meta-model; who uses it, what it is (describe/define), why is it important, where, when, how is it used
This system is built around a common ontology that needs must be understood by any system developers.  It is currently used to define relationships and entities within the system at design and run time.  The primary elements in this ontology are:
\begin{itemize}
\item \textit{Producer} This is a data producer as defined in our user model.  A data producer owns a given \textit{record} that has been created over a lifetime of medical care.
\item \textit{Consumer} Again from the user model, a data consumer.  Data consumers use medical data in some way.
\item \textit{Record} A medical record.  We can envision this as a set of discrete medical facts.
\item \textit{Filter} A transformation of a medical record.  If we have a record $ r $, we can transform $ r $ into $ r' $ by applying a transformation $ t $ such that $ r' = t(r) $ where $ t : record \rightarrow record $ and $ r' \subseteq r $.
\item \textit{Filtered Record} A filtered record is a record to which a filter has been applied.  If we have a filtered record $ r' $ derived from a record $ r $, then $ r' \subseteq r $.
\item \textit{License} A license describes the usage policy associated with a given filtered record.  This controls all aspects of filtered record use by an associated consumer.  The specific terms are negotiated over by the producer and the consumer until some consensus is reached, and they then bind the use of an associated filtered record.  Licenses must provide the ability to trace use of transitively associated artifacts regardless of the degree of separation as well.  For example, if we have an artifact $ a $ composed of sets of data elements $ e_{0}, e_{1}, ... , e_{n} $ derived from records $ r_{0}, r_{1}, ... , r_{n} $, we need to be able to ensure that any use of a set of data elements $ e_{i}, i < n $ is within the policy bounds of record $ r_{i}, i < n $ and any compensation associated with such use is correctly attributed to the original data owners and brokers.
\item \textit{Bundle} A filtered record and associated license.  This is distributed to data consumers.
\end{itemize}

\begin{figure}[!t]
\centering
\includegraphics[width=3in]{ontology}
\caption{System Ontology}
\label{System Ontology}
\end{figure}

\subsection*{Dynamic and Static Policy Evaluation}
%what who when where why how
%(what) define dynamic and static policy evaluation, elucidate specific advantages/disadvantages of each & where used, go over which we need to use here and why
Usage policies can be evaluated over a spectrum bordered by two distinct approaches - either dynamically, at request time, or statically, when a bundle is created.  Pure dynamic policy evaluation evaluates the entire policy against an artifact at \textit{request time}, specifically and only when a request for an action is made by a consumer.  Static evaluation only occurs when \textit{the bundle is created} and is not evaluated at any later time.  While dynamic policies are more powerful, static policies are generally simpler to define, create, and apply.  Dynamic policy evaluation requires significant runtime infrastructure as well, which static evaluation will never require.  Furthermore, that runtime infrastructure must be present in a variety of systems, implemented upon a myrad of platforms in a slew of different programming languages.  Still, we have compelling reasons for developing dynamic evaluation systems.  Static systems can't evaluate dynamic properties well.  Attributes like time are impossible to adjudicate with the simplest of static licenses and require some kind of dynamic evaluation.  Likewise, evaluation of a bundle's context is equally impossible to do with simple static policies.  Dynamic policies are more suitable for content that producers are interested in providing for unexpected use, while static policies generally only support predefined use scenarios.

In this system, we propose to use a combination of static and dynamic approaches.  Static policy evaluation occurs immediately after negotiation between the producer and consumer, when a filter is applied to the medical record.  This simplifies dynamic policy requirements by limiting the data that needs to be evaluated after the bundle is released.  If this filter were not applied, the dynamic policy would need to additional clauses to support hiding only those data elements to which the consumer has not been granted access.  All other evaluation occurs after the bundle is delivered to the consumer.  In order to support more complex and unenvisioned usage scenarios, including evaluating usage based on time constraints, this framework provides extensive dynamic evaluation capabilities after the initial filtering phase.  We also need to be able to support seamless operation over protected artifacts while disconnected from any kind of network or communication medium.  These factors lead to a powerful \textit{and local} dynamic policy evaluation system.

% An example of a floating figure using the graphicx package.
% Note that \label must occur AFTER (or within) \caption.
% For figures, \caption should occur after the \includegraphics.
% Note that IEEEtran v1.7 and later has special internal code that
% is designed to preserve the operation of \label within \caption
% even when the captionsoff option is in effect. However, because
% of issues like this, it may be the safest practice to put all your
% \label just after \caption rather than within \caption{}.
%
% Reminder: the "draftcls" or "draftclsnofoot", not "draft", class
% option should be used if it is desired that the figures are to be
% displayed while in draft mode.
%
%\begin{figure}[!t]
%\centering
%\includegraphics[width=2.5in]{myfigure}
% where an .eps filename suffix will be assumed under latex, 
% and a .pdf suffix will be assumed for pdflatex; or what has been declared
% via \DeclareGraphicsExtensions.
%\caption{Simulation Results}
%\label{fig_sim}
%\end{figure}

% Note that IEEE typically puts floats only at the top, even when this
% results in a large percentage of a column being occupied by floats.


% An example of a double column floating figure using two subfigures.
% (The subfig.sty package must be loaded for this to work.)
% The subfigure \label commands are set within each subfloat command, the
% \label for the overall figure must come after \caption.
% \hfil must be used as a separator to get equal spacing.
% The subfigure.sty package works much the same way, except \subfigure is
% used instead of \subfloat.
%
%\begin{figure*}[!t]
%\centerline{\subfloat[Case I]\includegraphics[width=2.5in]{subfigcase1}%
%\label{fig_first_case}}
%\hfil
%\subfloat[Case II]{\includegraphics[width=2.5in]{subfigcase2}%
%\label{fig_second_case}}}
%\caption{Simulation results}
%\label{fig_sim}
%\end{figure*}
%
% Note that often IEEE papers with subfigures do not employ subfigure
% captions (using the optional argument to \subfloat), but instead will
% reference/describe all of them (a), (b), etc., within the main caption.


% An example of a floating table. Note that, for IEEE style tables, the 
% \caption command should come BEFORE the table. Table text will default to
% \footnotesize as IEEE normally uses this smaller font for tables.
% The \label must come after \caption as always.
%
%\begin{table}[!t]
%% increase table row spacing, adjust to taste
%\renewcommand{\arraystretch}{1.3}
% if using array.sty, it might be a good idea to tweak the value of
% \extrarowheight as needed to properly center the text within the cells
%\caption{An Example of a Table}
%\label{table_example}
%\centering
%% Some packages, such as MDW tools, offer better commands for making tables
%% than the plain LaTeX2e tabular which is used here.
%\begin{tabular}{|c||c|}
%\hline
%One & Two\\
%\hline
%Three & Four\\
%\hline
%\end{tabular}
%\end{table}


% Note that IEEE does not put floats in the very first column - or typically
% anywhere on the first page for that matter. Also, in-text middle ("here")
% positioning is not used. Most IEEE journals/conferences use top floats
% exclusively. Note that, LaTeX2e, unlike IEEE journals/conferences, places
% footnotes above bottom floats. This can be corrected via the \fnbelowfloat
% command of the stfloats package.



\section*{Conclusion}
Evaluate results and outline future work

% conference papers do not normally have an appendix


% use section* for acknowledgement
\section*{Acknowledgment}
The authors would like to thank ECE, Ravi, Greg, Pramod?


% trigger a \newpage just before the given reference
% number - used to balance the columns on the last page
% adjust value as needed - may need to be readjusted if
% the document is modified later
%\IEEEtriggeratref{8}
% The "triggered" command can be changed if desired:
%\IEEEtriggercmd{\enlargethispage{-5in}}

% references section

% can use a bibliography generated by BibTeX as a .bbl file
% BibTeX documentation can be easily obtained at:
% http://www.ctan.org/tex-archive/biblio/bibtex/contrib/doc/
% The IEEEtran BibTeX style support page is at:
% http://www.michaelshell.org/tex/ieeetran/bibtex/
%\bibliographystyle{IEEEtran}
% argument is your BibTeX string definitions and bibliography database(s)
%\bibliography{IEEEabrv,../bib/paper}
%
% <OR> manually copy in the resultant .bbl file
% set second argument of \begin to the number of references
% (used to reserve space for the reference number labels box)
%\begin{thebibliography}{1}
\bibliographystyle{plain}
\bibliography{emr,drm}

%\bibitem{IEEEhowto:kopka}
%H.~Kopka and P.~W. Daly, \emph{A Guide to \LaTeX}, 3rd~ed.\hskip 1em plus
%  0.5em minus 0.4em\relax Harlow, England: Addison-Wesley, 1999.
  
%\bibitem{Jamkhedkar:Heileman}
%P.A.~Jamkhedkar and G.~L. Heileman, \emph{An Interoperable Usage Management Framework},
%  ACM DRM 2010, Chicago Ill, USA, 2010.

%\end{thebibliography}




% that's all folks
\end{document}


